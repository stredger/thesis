\startchapter{Experiments and Evaluation}
\label{chapter:exp}

%\section{Overview}

In this Chapter we examine the performance of Sage as well as how Sage performs in terms of the design goals. Specifically we look at the overhead of filesystem calls from within Sage versus performing the same calls outside of Sage. File reads and writes are measured to see the overhead on file operations, and file lists, removes, and creates are measured to see the overhead on filesystem operations. All measurements are made with the current implementation using two backends Swift and MongoDB. Furthermore, all experiments were performed on Emulab as I wanted the results to be as reproducible as possible \cite{emulab}.

We then look at a case study of an application using Sage to perform analysis on human and viral genomes specifically looking at how an application can take advantage of file placement within Sage. Finally, we conclude with an examination of types of applications that could take advantage of what Sage offers, and those that would be better off using a different system.

In the spirit of reproducibility, all the benchmarks are scripted and can be found at https://github.com/stredger/sagebench, while the case study can be found at https://github.com/stredger/dnasearch. Scatterplots for all data presented in this section can be found in Appendix \ref{chapter:appendix}.


\section{Microbenchmarks}
\label{sec:microbench}

I ran microbenchmarks on the Emulab computing platform using an Ubuntu12 64
bit OS image. Emulab allows us to run on bare metal, not inside a VM so we can
ignore any artifacts a VM may produce. I wanted to see how much overhead was
incurred by going through Sage instead of directly accessing files from their
respective backends. Additionally I wanted to see the differences between the
two implemented backends Swift and MondoDB. For the tests both MongoDB and
Swift were set up on the Emulab experiment node. Mongo was set up as a single
node configuration, while Swift had a proxy node and a storage node. I had to
use two nodes for Swift as using one node for both tasks was causing crashes
when bombarded with 10MB files. I used Emulabs d710 machines which are 64bit
and have 2GB of memory.

I ran tests by writing a simple Python script that performed file uploads and
downloads using the sync() and open() calls from SageFS. The put\_object() and
get\_object() calls were used from swiftclient to measure interaction with
Swift, and db.collection.insert() and db.collection.find\_one() from the
pymongo module to interact with MongoDB. These library calls are used
internally by Sage, so they can be used to measure interaction with just the
client versus the time to go through Sage. Timestamps are taken just before
the calls and just after they return, and stored in a list which is written to
after the experiment has completed. I ran each test $100$ times for a range of
file sizes $1kb$, $10kb$, $100kb$, $1mb$, $10mb$. I also attempted to run a
$100mb$ test but unfortunately MongoDB imposes an arbitrary file size limit of
$16mb$ so I did not run the $100mb$ test using the MongoDB backend. I could
have split up the file into smaller chunks, and in fact this is what MongoDB
recommends for large files, however this was not done. I made the decision not
to perform the test as MongoDB is clearly not suited for those types of files.
Additionally the test is meant to show the overhead of going through sage,
specifically if there are any points where Sage becomes an impediment over
normal usage, and in this case normal usage is to upload ten $10mb$ files
which is covered by another test. The test setup I used had a small disk size
for the swift backend so the $100m$ tests that used swift had to be performed
slightly differently.  The $100$ iterations had to be split up into runs where
$10$ files are uploaded, then have to be deleted to free up disk space so the
next run can proceed.

The platforms I measured were Swift, MongoDB, Sage using a Swift backend, Sage
using a MongoDB backend, and the local disk. I chose these to look at the
performance overhead of going through Sage compared to Swift and Mongo, and
the difference compared to using the local disk.

\note{swift used loopback on ext? might want to mention}


\input chapters/5/table.tex

Tables \ref{tab:microput} and \ref{tab:microget} summarize the micro
benchmark test results for all file sizes and backend stores.



\subsection{File Put Benchmarks}

\begin{figure}[h]
\centering
\resizebox{.9\linewidth}{!}{
\includegraphics{figures/plots/micro/all/multiput}
}
\caption[File Put Multiplot]{Multiplot for file Put times showing Median, Mean, Max, and Min times}
\label{fig:multiput}
\end{figure}


Figure \ref{fig:multiput} shows the results for all platforms to upload, or
`put', a file with the write() call into their respective backends. We use the log
of the file upload time so we can see the overall trend as the file size is
increased by an order of magnitude for each test. We look at the median of
each value to get a good representation of the upload time as outliers tend to
skew the mean and, being deterministic, computer measurements tend to clump in
stratifications. The median is an attempt to use the most common
stratification to represent the test result. Furthermore the min times are
quite similar to the median times which implies the mean is skewed by some
significant outliers, which is shown with the max times.


The local disk had the lowest put times for all file sizes, followed by
MongoDB, then Sage using Mongo, Swift, and finally Sage using Swift. The local
disk trend shows a fairly consistent increase in time as we increase the
filesize. Taking a closer look at the local test Figure
\ref{fig:locallogpointput} shows all 100 runs of each file size, and a nice
increase in upload time as file size increases. There is quite a lot of
variance in the points, especially at the 100MB test, however this is not
surprising as some runs will have cache misses and buffer flushes while others
will operate smoothly. Most tests follow the same trend An interesting anomaly
is for files of size 1k, where for all tests involving Swift the 1k times are
equal to or higher than the 10k file times. This could be due to the file
buffer in Python which batches I/O operations, however this uses the default
glibc buffer size (which is this case is the 8k buffer defined by BUFSIZ in
stdio.h and verified by a quick ptrace shown in Figure \ref{fig:writeptrace}
in the appendix) so unless I/O operations are buffered beyond the 8k by the
system, I/O buffering is not a likely culprit. If the buffer size was a
problem however it could easily be increase by passing an optional parameter
to Python's builtin open() which specifies the buffer size.


\begin{figure}[h]
\centering
\resizebox{.9\linewidth}{!}{
\includegraphics{figures/plots/micro/local/pointlogput}
}
\caption[Local Put Scatterplot]{Scatterplot of all times to put files locally. $100$ runs were performed for each filesize.}
\label{fig:locallogpointput}
\end{figure}


The larger times using Sage over the direct backend stores could be due to
having to copy the files contents into a SageFile object, which in this case
was an in memory file, then upload, as opposed to just uploading the contents
directly. This is the most likely the bulk of the additional time as the tests
are very similar except a few extra function calls in Python as Sage calls the
same client the direct tests used.

The median MongoDB times are slightly higher than the local times, however
Mongo has the largest max time for 10mB files. This could be due to MongoDB
filling up and having to extend the storage area as MongoDB pre allocates
files for collections with a default size and must grow the file as it runs
out of room. Additionally MongoDB indexes using B-Trees so the high max times
could be when the index has to grow the B-Tree, however splitting a B-Tree
leaf node most likely doesn�t add as much overhead as allocating new space so
it is not a likely cause.

SageFS using Mongo had slightly larger median upload times than just Mongo
itself, which is to be expected. Sage using Mongo did however have some of the
largest max times of all the measurements. It obviously suffers the same
problems as just plain Mongo, but also the was the Filesystem is implemented
causes some additional overhead. SageFS stores files in Mongo based on the
filename and path, while internally in Mongo every record is assigned a unique
id. SageFS must first check if the path name exists within Mongo, then if it
does it modifies the record, else it creates a new one. This is done as if we
simply try to upload the same path, a new record will be created with the same
path name (which is our key) and we end up with duplicate records. Normally
this is handled quite efficiently by Mongos internal indexing, however in the
worst case it could add some overhead. If the overhead of the file lookup
becomes a serious bottleneck the id could be generated by hashing the path
name with a collision resistant hash function, however there is always the
risk of a collision which would cause two files to be mapped to the same id!
Additionally MongoDB stores data in BSON, which is an extension to JSON which
can handle binary data. When we upload data into Mongo it must first be
encoded into BSON, which may add noticeable overhead for larger files.

Swift times were consistently the slowest over all the runs, however the Swift
setup was the only backend that used two machines so communication between the
two comes into play. Regardless the test are not to show a comparison between
Swift and Mongo, rather to show a comparison between using Swift or using
SageFS with a Swift backend. All the measurements for Swift were quite
consistent with each plot Median, Mean, Max, and Min having the same shape.
One thing to note is that the time to put 1k files is larger than 10k, 100k
and even 1mB files. Figure \ref{fig:swiftlogpointput} shows a scatterplot of all
Swift measurements. Surprisingly we actually see that the variance is quite
small for 1k files as well, which means the larger times are most likely not
due to random network fluctuations. Regardless the Swift setup I used has
problems with smaller files, either the actual file placement by Swift or the
files on the underlying xfs filesystem that Swift uses.

\begin{figure}[h]
\centering
\resizebox{.9\linewidth}{!}{
\includegraphics{figures/plots/micro/swift/pointlogput}
}
\caption[Swift Put Scatterplot]{Scatterplot of all times to put files into Swift. $100$ runs were performed for each filesize.}
\label{fig:swiftlogpointput}
\end{figure}



\subsection{File Get Benchmarks}


\begin{figure}[h]
\centering
\resizebox{.9\linewidth}{!}{
\includegraphics{figures/plots/micro/all/multiget}
}
\caption[File Get Multiplot]{Multiplot for file Get times showing Median, Mean, Max, and Min times}
\label{fig:multiput}
\end{figure}

The second microbenchmarks measured times to get files using the read() call
or the appropriate call to the backent that downloads file data into Sage. The
plots look very similar to the put times, we still have the local disk having
the lowest times and backends using Swift have the highest times. SageFS tests
using Swift and MongoDB backends were slightly slower than using their
backends alone. We do see that Sage using Mongo had very high max times, but
this time Mongo by itself did not show the same increased times. This may be
due to reading the returned data into a SageFile as the contents must be
decoded from BSON so they can be placed into a SageFile. But could also just
be an outlier in connecting to Mongo.



\section{Scalability}
\label{sec:scaleability}

The second set of benchmarks were testing the scalability of Sage and the
backend stores Swift and MongoDB. I also wanted to test how Sage performed
under conditions where Sage chose to place the file. To do this I ran a test
(sagerandom) where Sage chose to place the file randomly in either Swift or
Mongo compared to the other tests involving Sage explicitly requested to place
the file in Swift or Mongo. The test measured times to create, list, and
remove files as the number of files increases in the backend. The test
successively added (or removed) 1kB files up to 10000, so the first iteration
had zero previously existing files while the last had 9999. Unfortunately
since I did my testing on emulab a limit of 16 hours was imposed on the
experiment, therefore I could only gather 10 iterations of each run within the
time. I could have run more iterations over multiple experiment times (or in
parallel), but I wanted to have results from the same experiment that could be
easily reproduced.


\subsection{Listing Files}

The median times to list files is shown in Figure \ref{fig:scalelistmedian}. We can
see that Mongo took a very short amount of time to list compared to the other
backends, and also that it was consistent over all 10000 iterations. Sage
using a Mongo backend however took the largest amount of time and we can see
the first iteration was larger than just using Mongo. First the way the test
was set up, list() actually connected to both backends (Swift and Mongo) that
existed in the test setup. Even though there were no files present in the
Swift repository when we were performing the Mongo test, we still had to
connect to Swift and get back a list on zero files. This explains why both
Sage tests have similar times for zero existing files, which seems to be
limited by Swift. And second for the tests using Sage, the python objects
returned from Swift and Mongo client libraries have to be manipulated to
return reasonable paths for Sage. For both Mongo and Swift returned objects we
have to manipulate a list of dictionaries to extract the path name from other
data. In MongoDB�s case we actually return the entire record including data
which adds overhead, and furthermore in Mongo we have to concatenate the file
name to the path name stored in the Mongo record to return the full path we
want when listing files. Combined, the Python manipulation actually makes Sage
using Mongo have the largest increase in time as the number of files grows by
a large margin.

\begin{figure}[h]
\centering
\resizebox{.9\linewidth}{!}{
\includegraphics{figures/plots/scale/all/listmedian}
}
\caption[Median file list times]{Median times to perform list for each test run. Each test was performed $10$ times with an increasing number of files already existing from $0$ to $999$.}
\label{fig:scalelistmedian}
\end{figure}

Sage using Swift and Swift by itself showed fairly consistent results with
each other. Sage times were slightly larger, most likely due to the
aforementioned python object manipulation. Swift did have some anomalies were
times increased for a while then became stable again (stable meaning following
the trend visible in the plot). Most likely these are due to Swift getting
overwhelmed while flushing data to its underlying xfs filesystem, or clearing
some cached data somewhere. Interestingly such trends were not visible in Sage
using Swift so either the number of iterations was low enough that we simply
did not encounter the anomalies in the Sage runs, or the time between each
iteration was sufficient enough that we did not overload Swift. The only other
noticeable points are the high times with zero files present. This could be
due to caching issues within Swift as all times had variability and the zero
times were by no means the highest shown in Figure \ref{fig:scaleswiftlistscatter}, or
simply an artifact of the small number of runs.

\begin{figure}[h]
\centering
\resizebox{.9\linewidth}{!}{
\includegraphics{figures/plots/scale/swift/listmedian}
}
\caption{Scatterplot for list times in Swift.}
\label{fig:scaleswiftlistscatter}
\end{figure}

Finally we take a look at Sagerandom which is where files were randomly
assigned to the Swift or Mongo backends. Interestingly we see that the times
are essentially in between Sageswift and Sagemongo. Since approximately half
the files are in each backend it makes sense that placing the files randomly
essentially splits the difference and ends up approximately halfway in
between. This shows the scaling is tied to the backends and unsurprisingly the
overhead of having the filesystem naively choose the location is
insignificant.


\subsection{Creating Files}

The results of creating files for each test are summarized in Figure
\ref{fig:scalecreatemedian}. The times to create files had Sageswift with the
largest times, and Mongo with the lowest. Again Sagemongo shows the increase
in times while Mongo does not. We see something similar in the microbenchmarks
where we discuss how Sagemongo must search for existing files before it
creates a new one. Here we see the cost of searching for these files increase
as more are present in the backend. All other create times did not increase as
the number of files increased. Swift is relatively stable but again we see the
bumps not present with Sageswift. Like before, this is most likely overloading
Swift, although not very much as the time increase is quite small, or
variation we missed in Sageswift with the small number of runs. Since we do
see it in all the Swift plots the former is a likely culprit, however it is
not known for certain. Sageswift was by far the slowest, with median times
about three times larger than normal Swift. Previously I had argued that Sage
needed to read files into a SageFile (and therefore an extra buffer) to upload
to the desired backend. This is true but if we look at Sagemongo, it also has
to move data into a SageFile so we can not attribute all the overhead to
SageFiles. I also argued that Swift had trouble with small 1kB files, which
again we use here, however we would expect Swift and Sageswift to be closer if
these two were the only factors influencing the create time. When we create a
file in Sageswift we call open() on SageFS, under the hoods we first try to
download the file from Swift to check if it exists so we don�t accidentally
overwrite it. If the file doesn�t exist an exception is thrown, which we then
catch and can now safely create the file. This means that to create a file we
must talk to Swift twice (much like Sagemongo talks to Mongo twice) and do
some extra processing when creating a file in Sageswift compared to vanilla
Swift. The good news is that the overhead of talking to Swift is fixed and
should become a smaller portion of the overall time the larger the file
becomes. However this does mean that communicating with Swift (or Mongo) is
slow in the first place, the slow response will be amplified in create calls.
Finally we look at Sagerandom, which has three stratifications. We can see two
of them correspond to Sageswift and Sagemongo, while there is one sitting in
the middle of the two. Since files are randomly placed in either Swift or
Mongo we would not expect to find a median value where no value exists,
however since we use an even number of iterations (10) if half of the files
are using Swift and half using Mongo the mid point will be a split between
Sagemongo and Sageswift which is precisely what we see.

\begin{figure}[h]
\centering
\resizebox{.9\linewidth}{!}{
\includegraphics{figures/plots/scale/all/createmedian}
}
\caption[Median file create times]{Median times to create a file for each test run. Each test was performed $10$ times with an increasing number of files already existing from $0$ to $999$.}
\label{fig:scalecreatemedian}
\end{figure}


\subsection{Removing Files}

Finally we take a look at remove file times, which measures the time to remove
a file as an increasing number of files are present in the backend. The test
got a list of all the files present in the desired store then removed them one
by one measuring the time after each removal. The results are summarized in
Figure \ref{fig:scaleremovemedian}. Again we see our Mongo set up being
quicker than our Swift, but this time Mongo scales with the number of files
present, while Swift does not. Sagemongo scales slightly worse than Mongo by
itself which is somewhat surprising as on the client side Sagemongo only does
an extra string split and a few function calls than Mongo by itself. There is
a difference in how data is stored however, as Sagemongo stores file path and
name separately while directly using Mongo just stores the full path and
filename as one entity. We see another bump in the Swift times and again do
not see one in Sageswift. We do however see some variation in Sagerandom with
a large amount of files which may have been a bump, but it would be incorrect
to assume so with the amount of variation we see. Sageswift and Swift have
very similar performance with Sageswift looking slightly lower than Swift,
which this can be attributed to how I wrote the performance test where normal
Swift has to do a dictionary lookup while Sageswift does not. We can see
Sagerandom follows Sageswift (we read backwards as the first iteration
actually had $1000$ files present) until the midpoint, then there are a few
points in the middle which are the medians when exactly $50\%$ of the points
were in the Swiftbackend, and finally follows Sagemongo. Again this shows
scaling exactly like the backend it is using. The reason we see the three
stratifications separated the way they are come from the way list actually
returns results. The test simply iterated over the results returned from
list() and since each backend responds separately the files in Swift were
first in the list followed by Mongo.

\begin{figure}[h]
\centering
\resizebox{.9\linewidth}{!}{
\includegraphics{figures/plots/scale/all/removemedian}
}
\caption[Median file remove times]{Median times to remove a file for each test run. Each test was performed $10$ times with an increasing number of files already existing from $0$ to $999$.}
\label{fig:scaleremovemedian}
\end{figure}

Overall the scalability of Sage is mostly tied to the backend, or more
specifically the backend implementation done in Sage. The differences we see
are overheads caused by having to check if files exist which is the case for
create(), or format data which is the case for list(). The most drastic
difference is how SageMongo scales for list() compared to Mongo itself. List()
was the only call that scaled for all tests as file size increased. Create()
and Remove() were quite static for Swift backends. Mongo create was static but
SageMongo did increase in time as more files were present, while both tests
using Mongo have an increase in remove time as the number of files increases.


\section{Application Case Study}

\todo{ugh, make some subsections!!!!!}

In this section we look at a case study to show SageFS in use and how knowing
file location can be taken advantage of inside an application. The application
aligns viral DNA sequences to the human genome using the sequence alignment
tool Bowtie2. Aligning multiple virus to human genomes is embarrassingly
parallel. We want to compare viral genomes to ours, so naturally each virus
comparison doesn�t care about the state of the other comparisons. Knowing this
we can break up our job into many sub jobs and run them independently on
several machines, which makes it easy to construct a distributed system to
perform the alignments. The experiment environment uses the Geni Experiment
Engine to provide three nodes for computation 
\todo{cite{Berman20145}cite{GEEPAPER???} also Bowtie2 and Fabric, and NCBI, and SAVI}.
The nodes are controlled through a master using the Fabric
python module. Fabric essentially allows shell commands to be run on remote
machines through a python interface, and makes managing remote machines very
simple. I use SageFS to manage files across the GEE nodes, with Swift backends
on three clusters of machines on the Savi research network. The Sage backends
are located across Canada in Victoria, Toronto, and Carlton, while the GEE
nodes are located across the US in Utah, Illinois, and Maryland.

To get the virus data I wrote a simple web crawler in python that crawled the
NCBI web interface and downloaded all virus genomes, then stored them in
SageFS. Since the filesystem is already physically partitioned into three
locations, I decided to have three crawlers running in parallel, one on each
of the GEE nodes, and distribute the work so that approximately one third of
all the genomes would be processed each location. Each node was assigned a
location then when writing a file to Sage, requested the file be placed in the
assigned location. The end result was that Toronto ended up housing $1864$
genomes, Carlton $1729$, and Victoria $1942$. The split was not entirely
symmetrical as the work was partitioned on the number of links to follow, not
the genomes themselves, and any given link could contain a genome, multiple
genomes, or none at all.

To perform the experiment I wrote a simple python script that would grab a
list of all the virus genomes at a given location (either Victoria, Toronto,
or Carlon in this case), then iterate through downloading a genome and
aligning it against a reference index using Bowtie2. Each node had to have a
local copy of the reference index which the script grabbed from the Bowtie2
sourceforge site. I could have stored the reference in Sage, but the reference
index is $3.5$ GB and there was not enough space in the Swift repositories to
store it. Normally $3.5$ GB would not be a problem but since Sage is a
prototype and the machines are shared (Savi runs OpenStack, so the Swift
backends were inside VMs) I wanted it to have a minimal footprint on the
machine, so I did not use a large amount of disk space.

After the reference had been downloaded the node could start processing. For a
given virus genome the node would grab it out of Sage and transfer it to the
local filesystem (a function provided with Sage). This was done as the Sage
client is a python module and Bowtie2 is a standalone binary which uses the
system read and write calls. It is possible however to directly use Sage on
the local system using FUSE but this feature remains unimplemented and is
discussed as potential future work in Chapter \ref{chapter:conc}. After the
genome was local to the machine Bowtie2 was run with default local scoring
parameters looking for a local alignment comparing the virus sequence to the
human sequence, then the output file was placed back into Sage. I chose to
upload results back into the same location the original viral sequence was
from, however this decision was only made in an attempt to even the load at
each Swift repository as I was guaranteed that while an upload was taking
place at a given location, a download was not occurring at the same time. This
was in an effort to reduce the amount of time to perform all the alignments.
The physical location It took upwards of $36$ hours to to align all $5534$
sequences and produce the results. Only $36$ probable alignments were found,
but the actual results are unimportant. What is important is that the
application was able to take advantage of the physical location of files in
Sage, and process based on that information. Additionally the application was
able to choose where to upload the results which enabled the application to
distribute the load across the filesystem backends.

This was an ideal application to show off the file placement features of Sage,
and it worked very well for its purpose. In the next section we will take a
look at types of applications where Sage would be very beneficial, and
applications where Sage may not be the best choice.


\section{Application function and Sage}

The architecture of Sage allows applications to easily access files on a
common API regardless of the actual API of the backend store. This allows us
to build applications that can use a robust backend store or many, but stay
ignorant of the underlying API. Furthermore Sage exposes the physical location
of the backends which allows applications to take advantage of file location
and file placement. There are many types of applications that could take
advantage of Sage�s features. In this section we take a look at a few features
of applications that would benefit from using Sage, and others that would not
benefit from Sage.

\subsection{Leveraging Sage}

One type of applications that the current implementation of Sage makes a
suitable filesystem for are embarrassingly parallel apps. Sage works very well
with DropBox like functionality where files essentially live independently of
other files. This may seem like a trivial application, but it is used quite
heavily. Sage abstracts away the details of dealing with underlying stores so
applications can aggregate backends into a single resource, and access files
the same way regardless of where they are actually stored. In the genome
searching application this was extremely useful when parsing results. The
result files were uploaded into a specific location, however when parsing the
results to actually find matches I wrote a simple python script that iterates
over all sam output files (the file format output by Bowtie2) by calling
fs.list(). List with no arguments returns all files in the filesystem
regardless of location, so to the result parser the different locations were
simply different directories. This highlights another aspect of Sage where
applications that write many independent files can do so easily. The way I
wrote the genome searching app the result of an alignment is uploaded to the
same backend the original viral genome file came from. This allows
applications to write to the same filesystem, but have the load distributed
over multiple sites. The application could have also let the filesystem decide
where to place the files, again distributing the load across the backend
sites.

Another type of application that can benefit from Sage is one that uses read
only files. Sage allows file access from remote locations via client REST
calls. If data is never written to an opened file then Sage never runs into
any consistency issues and multiple readers can have access to the same file.
In the genome searching application the human reference genome indexes could
demonstrate this functionality within Sage as each remote site needs an
identical copy of the reference and can grab it from the filesystem directly,
instead of having to pull it from a web site. This may seem like the same task
but consider that interacting with Sage uses posix like filesystem commands,
while downloading something from off of the web requires an HTTP client. The
application now has to communicate with two separate protocols, while using
Sage would limit to just one. Additionally if Sage had a FUSE implementation,
applications could mount a Sage instance and read and write remote files using
normal system commands.

Applications that can take advantage of the exposed physical location of files
can also benefit from using Sage. In the genome searching application I
partitioned the viral genome files placing approximately a third at each of
three physical locations. I was able to then partition the alignment
computation based on the physical partitioning of the files. Now consider the
Greencities application from Chapter \ref{chapter:intro}. If the application stored images
in Sage at multiple backends and had distributed nodes in different locations,
using Sage would allow the application to essentially move the computation to
the closest node to reduce file transfer times. If for example the application
had the same backend Swift stores as the genome searching application, and had
compute nodes in Seattle and New York, Sage would allow the compute nodes to
work on the closest subset of satellite images, simply by partitioning on path
name, which to the application looks like sub directories. Another example of
Sage benefiting location sensitive applications is when sensitive data is
partitioned across Canada and the US. Consider an application that handles
financial records for a multinational bank. Canadian records must be stored in
Canada while US records must reside in the US. Using Sage the application can
safely place sensitive files in the correct physical location, while still
being able to access all files as part of a larger filesystem.

Finally consider a user with a collection of remote resources. An account on
cloud storage platforms Dropbox, Google Drive, and Amazon S3. Suppose the user
has limited capacity on each cloud platform, but wants to store more files
than can fit on one platform. Using Sage the user can aggregate all their
cloud storage into a single filesystem. This is especially useful in personal
or smaller cloud environments where physical resources are not especially
abundant.

\todo{Cite this shit homie!}


\subsection{Burdened by Sage}

Sage is great for location aware applications or aggregating resources,
however the architecture makes it so Sage may be unsuitable for some types of
applications. Since Sage relies on the backend stores to handle things like
replication, metadata management, and file locking, if a backend store
mishandles or does not handle one of the features, Sage does not either. This
makes the current implementation of Sage unsuitable for applications that
write to the same file. Swift and Mongo provide no file locking mechanism so
if a distributed application opens a file at two different locations and
makes edits, the result will be the file that was written last (last write
wins). Therefore applications that reduce results down to a single file will
likely not want to use Sage. If for example in the genome searching
application I wanted to process my results in parallel, I would have to make
sure no process was writing to the results file at the same time. Additionally
each time I wanted to update the file, I would have to make sure I had the
latest copy. Unfortunately this is a difficult problem to solve and as
discussed in Chapters \ref{chapter:arch} and \ref{chapter:conc} there needs to be some
guarantee of atomicity at some level. GFS gets around this by having an atomic
append operation, while other filesystems behave like Sage in that last write
wins.

Applications that rely heavily on performance will also struggle using Sage.
We saw earlier in the chapter in sections \ref{sec:microbench} and
\ref{sec:scaleability} that there is some overhead when dealing with Sage. The goal
of Sage is to aggregate remote storage together and to expose physical
location to the application, not raw performance. If an application heavily
relied on the performance of the backend store, the app should directly access
the backend rather than go through Sage.
