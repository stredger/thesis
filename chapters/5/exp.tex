\startchapter{Experiments and Evaluation}
\label{chapter:exp}

%\section{Overview}

In this Chapter we examine the performance of Sage as well as how Sage performs in terms of the design goals. Specifically we look at the overhead of filesystem calls from within Sage versus performing the same calls outside of Sage. File reads and writes are measured to see the overhead on file operations, and file lists, removes, and creates are measured to see the overhead on filesystem operations. All measurements are made with the current implementation using two backends Swift and MongoDB. Furthermore, all experiments were performed on Emulab as I wanted the results to be as reproducible as possible \cite{emulab}.

We then look at a case study of an application using Sage to perform analysis on human and viral genomes specifically looking at how an application can take advantage of file placement within Sage. Finally, we conclude with an examination of types of applications that could take advantage of what Sage offers, and those that would be better off using a different system.

In the spirit of reproducibility, all the benchmarks are scripted and can be found at https://github.com/stredger/sagebench, while the case study can be found at https://github.com/stredger/dnasearch. Scatterplots for all data presented in this section can be found in Appendix \ref{chapter:appendix}.


\section{Microbenchmarks}
\label{sec:microbench}

I ran microbenchmarks on the Emulab computing platform using an Ubuntu12 64-bit OS image. Emulab allows us to run on bare metal, not inside a VM so we can ignore any artifacts a VM may produce. I wanted to see how much overhead was incurred by going through Sage instead of directly accessing files from their respective backends. Additionally I wanted to see the differences between the two implemented backends Swift and MondoDB. Both MongoDB and Swift were set up on the Emulab experiment nodes. Mongo used a single node configuration while Swift had one proxy node and one storage node. Swift needed two nodes as using a single node for both storage and proxy, two components Swift requires, was causing crashes when running the larger tests. For these microbenchmarks, I used Emulabs d710 machines that are 64-bit and have 2GB of memory.

I ran tests by writing a simple Python script that performed file uploads and downloads using the sync() and open() calls from SageFS. The put\_object() and get\_object() calls were used from the swiftclient Python module to measure interaction with Swift, and db.collection.insert() and db.collection.find\_one() from the pymongo module to interact with MongoDB. These library calls are used internally by Sage so are used to measure time to go through the module versus the time to go through Sage. Timestamps are taken just before a call and just after it returns. Timestamps are stored in a list that is written to disk after the experiment has completed. 

I ran each test 100 times for a range of file sizes 1KB, 10KB, 100KB, 1MB, and 10MB. I also attempted to run a 100MB test, but unfortunately MongoDB imposes an arbitrary file size limit of 16MB so I did not run the 100MB test using the MongoDB backend. I could have split up the file into smaller chunks, and in fact this is what MongoDB recommends for large files. However, I made the decision not to perform the test as normal usage in this case is to upload ten 10MB files, which is simply the 10MB file test with more runs.

The Emulab nodes had a small disk size, so the 100MB tests were performed slightly differently. The hundred iterations were split up into runs of ten. A run uploaded ten files, then deleted them to free up disk space so the next run could proceed. 

The platforms I measured were Swift, MongoDB, Sage using a Swift backend, Sage using a MongoDB backend, and the local disk. I chose these to look at the performance overhead of going through Sage compared to Swift and Mongo. The local disk is used as a measuring stick to put the measurements into context.

\todo{swift using loopback xfs}


\input chapters/5/table.tex

Tables \ref{tab:microput} and \ref{tab:microget} summarize the micro
benchmark test results for all file sizes and backend stores.



\subsection{File Put Benchmarks}

\begin{figure}[h]
\centering
\resizebox{.9\linewidth}{!}{
\includegraphics{figures/plots/micro/all/multiput}
}
\caption[File Put Multiplot]{Multiplot for file Put times showing Median, Mean, Max, and Min times}
\label{fig:multiput}
\end{figure}


Figure \ref{fig:multiput} shows the results for all platforms to upload, or `put', a file into their respective backends. We use the log of the file upload time so we can see the overall trend as the file size increases by an order of magnitude for each test. We look at the median of each value to get a good representation of the upload time. Outliers tend to skew the mean and, being deterministic, computer measurements tend to clump in stratifications. The median is an attempt to use the most common stratification to represent the test result. Furthermore, the minimum times are quite similar to the medians, which implies the mean is skewed by some significant outliers. The outliers are shown in the max times with some being orders of magnitude larger than the medians.

The local disk had the lowest put times for all file sizes, followed by MongoDB, Sage using Mongo, Swift, and finally Sage using Swift. The local disk trend shows a fairly consistent increase in time as we increase the file size. Taking a closer look at the local test, Figure \ref{fig:locallogpointput} shows all 100 runs of each file size. As expected, we see an increase in upload time as file size increases. We see a significant amount of variance in the plot, especially at the 100MB test. This is not surprising however as some runs have cache misses and buffer flushes causing delays while others operate smoothly. Most tests follow the same trend. An interesting anomaly is for 1KB files, where all tests involving Swift have 1KB times equal to or higher than 10KB file times. This could be due to the file buffer in Python, which batches I/O operations. However, python I/O uses the default glibc buffer size, which in this case is the 8KB buffer defined by BUFSIZ in stdio.h. The python buffer size is verified by a ptrace shown in Figure \ref{fig:writeptrace} in Appendix \ref{chapter:appendix}. Unless the system buffers I/O operations beyond Python's 8KB, I/O buffering is not a likely culprit. However, if the buffer size was a problem it can easily be increased by passing an optional buffer size parameter to Python's builtin open().

\begin{figure}[h]
\centering
\resizebox{.9\linewidth}{!}{
\includegraphics{figures/plots/micro/local/pointlogput}
}
\caption[Local Put Scatterplot]{Scatterplot of all times to put files locally. 100 runs were performed for each filesize.}
\label{fig:locallogpointput}
\end{figure}

The larger times we see using Sage over the direct backend stores could be due to having to copy the files contents into a SageFile object. Data is first copied into an in memory file, then upload, as opposed to just uploaded directly. This is the most likely the bulk of the additional time as the tests are very similar except a few extra function calls in Python as Sage calls the same client the direct tests used.


The median MongoDB times are slightly higher than the local times; however MongoDB has the largest max time for 10MB files. This could be due to MongoDB filling up and having to extend the storage area as MongoDB pre-allocates files for collections with a default size, and must grow the file as it fills up. Additionally MongoDB indexes data using B-Trees, so the high max times could be when the index has to grow the B-Tree. However, splitting a B-Tree leaf node most likely doesn�t add as much overhead as allocating new space so it is not a likely cause. Finally MongoDB stores data in BSON, which is a JSON extension for binary data. When we upload data into MongoDB, it must first be encoded into BSON, which may add noticeable overhead for larger files.


As expected SageFS using MongoDB had slightly larger median upload times than just MongoDB. Sage using MongoDB also has some of the largest max times of all the measurements. Obviously Sage using MongoDB encounters the same performance issues as just plain MongoDB, but also the implementation of the translator causes some additional overhead. SageFS stores files in MongoDB based on the filename and path, while internally MongoDB identifies every record by assigning a unique id. To write to a file, the MongoDB translator must first check if the requested path and filename exists within MongoDB. If the path exists the translator modifies the record, else it creates a new file at the specified path. If we simply try to upload the same path to modify a file without addressing by id, a new record will be created with the same path name, and we end up with duplicate records. File lookups  are handled quite efficiently by MongoDB's internal indexing, however in the worst case it could add noticeable overhead. For future implementations, if the file lookup overhead becomes a serious bottleneck an id could be generated by hashing the path name with a collision resistant hash function. However, there is always the risk of a collision that would cause two files to be mapped to the same id!


Swift times were consistently the slowest over all the runs, however the Swift setup was the only backend that used two machines so communication between the two comes into play. Regardless the test are not to show a comparison between Swift and Mongo, rather to show a comparison between using Swift or using SageFS with a Swift backend. All the measurements for Swift were quite consistent with each plot Median, Mean, Max, and Min having the same shape. One thing to note is that the time to put 1k files is larger than 10k, 100k and even 1mB files. Figure \ref{fig:swiftlogpointput} shows a scatterplot of all Swift measurements. Surprisingly we actually see that the variance is quite small for 1k files as well, which means the larger times are most likely not due to random network fluctuations. Regardless the Swift setup I used has problems with smaller files, either the actual file placement by Swift or the files on the underlying xfs filesystem that Swift uses. 

\begin{figure}[h]
\centering
\resizebox{.9\linewidth}{!}{
\includegraphics{figures/plots/micro/swift/pointlogput}
}
\caption[Swift Put Scatterplot]{Scatterplot of all times to put files into Swift. $100$ runs were performed for each filesize.}
\label{fig:swiftlogpointput}
\end{figure}



\subsection{File Get Benchmarks}


\begin{figure}[h]
\centering
\resizebox{.9\linewidth}{!}{
\includegraphics{figures/plots/micro/all/multiget}
}
\caption[File Get Multiplot]{Multiplot for file Get times showing Median, Mean, Max, and Min times}
\label{fig:multiget}
\end{figure}

The second microbenchmarks measured times to get files using read() or the appropriate backend call to download file data into Sage. The plots, shown in figure \ref{fig:multiget}, look very similar to the put times; we still have the local disk with the lowest times and backends using Swift with the highest. SageFS tests using Swift and MongoDB backends were slightly slower than using the backends without Sage. We do see that Sage using MongoDB had very high max times, but this time MongoDB by itself did not show the same large maximums. The MongoDB time discrepancies may be due to reading returned data into a SageFile as the contents must be decoded from BSON when placed into a SageFile. However, could also just be an outlier in connecting to MongoDB.


\section{Scalability}
\label{sec:scaleability}

The second set of benchmarks tested the scalability of Sage and the backend stores Swift and MongoDB. I also wanted to test how Sage performed when it made file placement decisions. To test random performance, I ran a test (sagerandom) where Sage chose to place a file randomly in either Swift or Mongo. For all other tests involving Sage, files are explicitly requested to be placed in either Swift or MongoDB backends.

The scalability tests measure times to create, list, and remove files as the number of files increases in the backend. The create test successively added up to 1000 1KB files, so the first iteration had zero previously existing files while the last had 999. I did the testing on Emulab that has an experiment limit of 16 hours. Unfortunately, I could only gather ten iterations of each run within the time. I could have run more iterations over multiple experiment times (or in parallel), but I wanted to have results from the same experiment that could be easily reproduced.

\subsection{Listing Files}

Figure \ref{fig:scalelistmedian} shows the median times to list files. We can see listing files in MongoDB took a very short amount of time compared to the other backends. Additionally MongoDB listing had little  variation over all 10000 iterations. Sage using MongoDB however took the largest amount of time. We can see the very first iteration, where only one file exists already, was larger than just using Mongo. Like most operations in Sage, when we list files we can directly address a backend or choose not to. In the list test, I called the list() method on the entire Sage filesystem so list() connected to both backends Swift and MongoDB. Even though there were no files present in the Swift repository, Sage still had to connect to Swift and get back a list on the empty backend. This explains why both Sage tests have similar times for zero existing files, which seems to be limited by Swift.

For tests using Sage, the objects returned from the Swift and MongoDB client libraries have to be manipulated to return reasonable paths for Sage. Sage manipulates a list of Python dictionaries returned from the backends to extract the path name from other data. In MongoDB�s case, Sage sees the entire record including data. So a list on MongoDB returns all files in the backend! This solution obviously does not scale with larger files; however it is an implementation detail of list() in the MongoDB translator. Furthermore with MongoDB, Sage has to concatenate two fields in a returned MongoDB record to construct a file path. This Python manipulation makes Sage using Mongo have the largest increase in time as the number of files grows, by a large margin.


\begin{figure}[h]
\centering
\resizebox{.9\linewidth}{!}{
\includegraphics{figures/plots/scale/all/listmedian}
}
\caption[Median file list times]{Median times to perform list for each test run. Each test was performed $10$ times with an increasing number of files already existing from $0$ to $999$.}
\label{fig:scalelistmedian}
\end{figure}

Sage using Swift and Swift by itself showed fairly consistent results with each other. Sage times were slightly larger, most likely due to the aforementioned Python object manipulation. Swift did have some anomalies where times increased, then became stable again (stable meaning following the trend visible in the plot). Most likely these are due to Swift getting overwhelmed while flushing data to its underlying xfs filesystem, or clearing some cached data somewhere. Interestingly, such trends are not visible in Sage using Swift. Either the number of iterations was low enough that I did not encounter the anomalies in the Sage runs or going through Sage allowed enough time for Swift to handle all requests without overloading. The only other noticeable points are the high median times with zero files present. This could be due to caching issues within Swift. Figure \ref{fig:scaleswiftlistscatter} shows that all times had variability, and the zero times were by no means the highest, so the very first run could be encountering more cache misses than others. As always though the times could be an artifact of the small number of runs.

\begin{figure}[h]
\centering
\resizebox{.9\linewidth}{!}{
\includegraphics{figures/plots/scale/swift/listmedian}
}
\caption{Scatterplot for list times in Swift.}
\label{fig:scaleswiftlistscatter}
\end{figure}

Finally, we take a look at Sagerandom. In this test, files were randomly assigned to the Swift or MongoDB backends. Interestingly we see the times are in between Sageswift and Sagemongo. Since approximately half the files are in each backend, it makes sense that placing files randomly essentially splits the difference and ends up approximately halfway in between. This shows the scaling is tied to the backends and unsurprisingly the overhead of having the filesystem naively choose the location is insignificant. More sophisticated file placement could incur more overhead depending on the complexity of the file placement function within Sage.

\subsection{Creating Files}

Figure \ref{fig:scalecreatemedian} summarizes the results of creating files for each test. Sageswift had the largest times, and MongoDB had the lowest. Again Sagemongo shows an increase in times while MongoDB does not. We see something similar in the microbenchmarks where we discuss how Sagemongo must search for existing files before it creates a new one. Here we see the cost of searching for existing files increase as more are present in the backend. All other tests create times did not increase significantly as the number of files increased.

Swift is relatively consistent but again we see bumps not present with Sageswift. Like before, this is most likely overloading Swift or again variation we missed in Sageswift with the small number of runs. Since we do see the bumps in all Swift plots overloading is a likely culprit however the exact cause remains unknown. 

Sageswift was by far the slowest, with median times about three times larger than normal Swift. Previously I argued that Sage needed to read data into a SageFile (and therefore an extra buffer) to upload to the desired backend. While this is true, Sagemongo also has to move data into a SageFile so we can not attribute all the overhead to SageFiles. I also argued that Swift had trouble with small 1KB files, which again we see here; however we would expect Swift and Sageswift to be closer if these were the only factors influencing the create time. To create a file in Sageswift, we call SageFS's open(), which is forwarded to the Swift translator. In the translator Sage first tries to download the file from Swift to ensure the file is not accidentally overwritten. The swiftclient module throws an exception if the file does not exist; which Sage then catches and can now safely create the file. So to create a file Sage must contact Swift twice much like Sagemongo talks to MongoDB twice. The good news is that the overhead of talking to Swift is fixed and should become a smaller portion of the overall time the larger the file becomes. However, this does mean that communicating with Swift (or MongoDB) is noticeable. Even more so with create calls as we incur the cost twice. 

Finally, we look at Sagerandom. The plot has three stratifications, two matching the Sageswift and Sagemongo plots closely while the third sits in the middle of the two. Since with Sagerandom files are randomly placed in either Swift or MongoDB, we would not expect to find middle values as no actual values exist between Swift and MongoDB. However, since I used ten iterations, if half of the files are using Swift and the other half using MongoDB, the median point will be a split between them, which is precisely what we see.

\begin{figure}[h]
\centering
\resizebox{.9\linewidth}{!}{
\includegraphics{figures/plots/scale/all/createmedian}
}
\caption[Median file create times]{Median times to create a file for each test run. Each test was performed $10$ times with an increasing number of files already existing from $0$ to $999$.}
\label{fig:scalecreatemedian}
\end{figure}


\subsection{Removing Files}

Finally, we take a look at remove file times, which measures the time to remove a file as an increasing number of files are present in a given backend. The remove test first gets a list of all files present in the desired backend then removes them one by one measuring the time after each removal. 

The results are summarized in Figure \ref{fig:scaleremovemedian}. Again we see the MongoDB instance is quicker than Swift, but this time MongoDB scales with the number of files present, while Swift does not. Sagemongo scales slightly worse than MongoDB by itself, which is somewhat surprising as on the client side Sagemongo only performs an extra string split and a few function calls than MongoDB by itself. However, there is a difference in how Sagemongo and MongoDB stores test data. Sagemongo stores file path and name separately, while directly using MongoDB stores the full path as one entity. I store the file path and name separately in Sagemongo as it is easier to support queries on paths with the two split.

We see another bump in the Swift times and again it is absent from Sageswift. We do, however, see some variation in Sagerandom with many files. This variation may be similar to the bumps we see in Swift, but it would be incorrect to assume so with the amount of variation we see. Sageswift and Swift have very similar performance, with Sageswift slightly lower than Swift. This could be attributed to how I wrote the scalability test. To test Swift I have to perform a dictionary lookup to get the correct path name while Sageswift is supplied the path already. I do this as the list call returns either a list of paths, such as with Sageswift, or the raw objects from the swiftclient library as done with Swift.

Sagerandom follows Sagemongo until the midpoint; then has a few points in the middle (the medians when exactly 50\% of the points were in each backend), and finally follows Sageswift. Again this shows scaling exactly like the backend the random part is using. The reason we see the three stratifications separated comes from the way list actually returns results. For the remove test we list all files. Since each backend responds separately, the returned list is ordered according to backend. The files in Swift were first in the list followed by MongoDB so from right to left we see files is Swift are removed first, followed by those in MongoDB.

\begin{figure}[h]
\centering
\resizebox{.9\linewidth}{!}{
\includegraphics{figures/plots/scale/all/removemedian}
}
\caption[Median file remove times]{Median times to remove a file for each test run. Each test was performed $10$ times with an increasing number of files already existing from $0$ to $999$.}
\label{fig:scaleremovemedian}
\end{figure}

Overall the scalability of Sage is mostly tied to the backend, or more specifically the backend translator implementation done in Sage. The differences we see are overheads caused by a few issues. Checking if files exist causes overhead for create() calls and formatting returned data causes overhead for list(). The most drastic difference in backends we see is how SageMongo scales for list() compared to MongoDB itself. list() was the only call that increased in time as file size increased for all tests. create() and remove() times were quite static for Swift backends. Mongo create() was static, but SageMongo increased in time as more files were present. Both tests using Mongo showed an increase in remove() time as the number of files increases. 

\section{Application Case Study}

In this section, we look at a case study to show SageFS in use and how applications can take advantage of file location. The application we examine aligns viral DNA sequences to the human genome using the sequence alignment tool Bowtie2 \cite{langmead2012fast}. The application wants to compare viral genomes to ours, so naturally each virus comparison is independent of all others. Knowing this fact, we can break up the computation into many sub jobs and run them independently on several machines. This partitioning makes it easy to construct a distributed system to perform the alignments. The experiment environment uses the Geni Experiment Engine to provide three nodes for computation \cite{gee2014}. Nodes are controlled through a master using the Fabric python module \cite{fabric}. Fabric allows shell commands to be run on remote machines through a python interface, which makes managing remote machines very simple. I use SageFS to manage files across the GEE nodes. The SageFS deployment had Swift backends on three clusters of machines on the Savi research network \cite{savi}. The Sage backends are located across Canada in Victoria, Toronto, and Carlton, while the GEE nodes are located across the US in Utah, Illinois, and Maryland.

I wrote a simple web crawler in python that crawled the NCBI web interface and downloaded all virus genomes \cite{ncbi}. The genomes were stored  in Sage. Since the Sage deployment was already physically partitioned into three locations, I decided to have three crawlers running in parallel. One crawler was deployed on each of the GEE nodes. I distributed the work so that each node would process approximately one third of all the viral genomes in the NCBI. Each crawler was assigned a location, which was used to place the downloaded genomes within Sage. After the database was crawled Toronto ended up housing 1864 genomes, Carlton 1729, and Victoria 1942. The genome split was not entirely symmetrical as the work was partitioned on the number of links to follow, not the genomes themselves. Any given link could contain a genome, multiple genomes, or none at all.

I wrote a separate Python script to perform the genome alignments. Using the Sage list call, the script grabs a list of all virus genomes at a given location (either Victoria, Toronto, or Carlon in this case). The list is then iterated through, opening each genome file locally and aligning it against a human genome reference index using Bowtie2. Each node has to have a local copy of the reference index which the script grabs from the Bowtie2 SourceForge site. Sage could store the reference index; however it is 3.5GB, and there was not enough space in the Swift repositories used to store it. Normally 3.5GB would not be a problem, but since Sage is a prototype, and the Savi machines are shared, I wanted to have a minimal footprint within the machines. Therefore, I did not create large disks for Swift to use in the backend.

After the reference is downloaded, a node can start processing. For a given virus genome, a node grabs it out of Sage and transfers it to the local filesystem (a function provided with Sage). I have to transfer to the local fileystem as the Sage client is a Python module, and Bowtie2 is a standalone binary which uses the system read and write calls. It is possible however to directly use Sage on the local system using FUSE, but this feature remains unimplemented and is discussed as potential future work in Chapter \ref{chapter:conc}. After the virus genome is local to a machine Bowtie2 is run with default local scoring parameters. Bowtie2 looks for a local alignment comparing virus to human sequence. One complete the output file is placed back into Sage. I chose to upload results back into the same location the original viral sequence was from. This decision was made in an attempt to even the load at each Swift repository. I am guaranteed that while an upload is taking place at a given location, a download is not occurring at the same time as each node is responsible for all sequence at only one location. It took upwards of 36 hours to to align all 5534 sequences and produce the results. Only 36 probable alignments were found, but the actual results are unimportant. What is important is that the application was able to take advantage of the physical location of files in Sage, and process based on that information. Additionally the application was able to choose where to upload the results enabling the application to distribute server load across the filesystem backends.

This application was ideal to show off the file placement features of Sage. The application can partition computation based on querying file locations within Sage, and distribute load across Sage backends when uploading files. In the next section, we take a look at types of applications that can take advantage of Sage's features, and those that can not.


\section{Application function and Sage}

The architecture of Sage allows applications to access files with a common API regardless of the actual API of the backend store. This feature allows us to build applications that can use an assortment of backends, but stay ignorant of the underlying API. Furthermore, Sage exposes the physical location of backends that allows applications to take advantage of file location and control file placement. There are many types of applications that could take advantage of Sage�s features. In this section we take a look at a few types of applications that would benefit from using Sage, and others that would not.

\subsection{Leveraging Sage}

Sage makes a very suitable filesystem for embarrassingly parallel applications. Sage works well with DropBox like functionality, where files live independently on a remote host. Sage abstracts away the details of dealing with underlying stores so applications can aggregate backends into a single resource, and access files the same way regardless of where they are stored. In the genome searching application, this feature was extremely useful when parsing results. Result files were uploaded into a specific location. However, when parsing the results to find matches, I wrote a simple python script that iterated over all .sam output files (the file format output by Bowtie2) by calling fs.list(). List with no arguments returns all files in the filesystem regardless of location, so the result parser saw different locations simply as different directories. 

Another benefit of using Sage is applications that write many independent files can do so easily. The way I wrote the genome searching app the result of a sequence alignment is uploaded to the same backend as the original viral genome. Applications can write to the same filesystem but have the load distributed over multiple sites. Furthermore, the application could have let the filesystem decide where to place result files. Again distributing load across backend sites, or distributing with respect to another factor such as latency, remaining backend space, or location to name a few.

Applications that use read only files can also benefit from using Sage. Sage allows file access from remote locations via client REST calls. If data is never written to an opened file, Sage never runs into any consistency issues, and multiple readers can have access to the same file. In the genome searching application, the human reference genome indexes is a great candidate for this functionality. Each remote site needs an identical copy of the reference and can grab it from the filesystem directly, instead of having to pull it from a different service. In both cases it may seem like just downloading a file, but consider that interacting with Sage uses posix like filesystem commands, while downloading something from the web requires an HTTP client. The application has to communicate with two separate protocols, the filesystem API and HTTP. Using Sage, the application only has to use the filesystem API. Additionally if Sage had a FUSE implementation, applications could mount a Sage instance then read and write remote files using normal operating system calls.

Applications that can take advantage of the exposed physical location of files can also benefit from using Sage. In the genome searching application, I partitioned the viral genome files placing approximately one third at each of three physical locations. I was then able to partition the alignment computation based on the physical partitioning of the files. Now consider the Green Cities application from Chapter \ref{chapter:intro}. Suppose the application partitioned images across multiple Sage backends, and had distributed nodes like the genome searching application. Using Sage would allow the application to move its greenspace computation to nodes closes to required images to reduce file transfer times. If for example, the application had the same backend Swift stores as the genome searching application, and had compute nodes in Seattle and New York, Sage would allow the compute nodes to work on the closest subset of satellite images. This  could be accomplished by partitioning computation on image path name, which to the application looks just like two directories of images. 

Another example application is one where sensitive data is partitioned across Canada and the US. Consider an application that handles financial records or school grades. For this application Canadian records must be stored in Canada while US records must reside in the US. Using Sage, the application can safely place sensitive files in the correct physical location while still being able to access all files as part of a larger filesystem.

Finally, consider a user with a collection of remote resources. An account on cloud storage platforms Dropbox, Google Drive, and Amazon S3. Suppose the user has limited capacity on each cloud platform, but wants to store more files than can fit on one platform. Using Sage the user can aggregate all their cloud storage into a single filesystem. This aggregation is especially useful in personal or smaller cloud environments where physical resources are not especially abundant.

\subsection{Burdened by Sage}

Sage is great for location aware applications or aggregating resources. However, Sage's architecture makes unsuitable for some types of applications. Since Sage relies on backends to handle things like replication, metadata management, and file locking, if a backend store mishandles or does not handle one of the features, Sage does not either. This design makes the current implementation of Sage unsuitable for applications that concurrently write to the same file. Swift and MongoDB provide no file locking mechanism. If a distributed application opens a file at two different locations and makes edits, the resulting file data will be the whatever was written last (last write wins). Therefore, applications that reduce results to a single file, will likely not want to use Sage. If, for example, in the genome searching application I wanted to process results in parallel, I would have to make sure no two processes were writing to the results file at the same time. Additionally each time I wanted to update the result file, I would have to make sure I had the latest copy. 

Unfortunately file locking and consistency are difficult problems to solve and as discussed in Chapters \ref{chapter:arch} and \ref{chapter:conc} there needs to be some guarantee of atomicity at some level to implement solutions. As seen in Chapter \ref{chapter:rel_work} GFS works around this by having an atomic append operation while other filesystems behave like Sage with last write wins semantics.

Applications that rely heavily on performance will also struggle using Sage. We saw earlier this Chapter in sections \ref{sec:microbench} and \ref{sec:scaleability} that Sage has some performance overhead. The goal of Sage is to aggregate remote storage together and to expose physical location to the application, not raw performance. If an application heavily relied on the performance of a backend store, the application should directly access the backend rather than go through Sage.
