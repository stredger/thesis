\startfirstchapter{Introduction}
\label{chapter:intro}

Distributed applications are becoming ubiquitous in our everyday lives. We
send mail with Gmail, communicate with friends and family through Facebook,
seek entertainment with Netflix, and get directions with Google Maps. These
types of applications take advantage of distributed storage to help with
issues including load balancing, parallel data access, durability,
accessibility, and size constraints.

Distributed storage can be leveraged by smaller scale applications as well. In
2012, a group of us at UVic decided to make an interesting application for the
upcoming Geni Engineering Conference that calculated the amount of green space
contained within a city. While the actual green space counting was a fun
result, the application was a demonstration of big data on a GENI environment
\cite{Geni}. Called Green Cities, we took satellite imagery that spanned the
entire globe and essentially counted pixels within city limits to find the
amount of green space. We had over 460GB of images that we stored in an
OpenStack Swift \cite{Swift} storage cluster. The application was extremely
parallelizable and allowed us to partition the computation over the nodes we
had. Each node needed access to all files, and since we did not have enough
space on a single Swift cluster, we had to spread data out over a number of
nodes. Eventually, we brought the experiment down to give other users access
to the resources we were using. So the next time we attempted to revive the
application all the systems, including the file storage, had to be set up
again. It was clear that having access to data in a globally accessible
filesystem would be a great asset to these types of experiments.

Experiments and applications use a wide array of storage devices. We used
Swift but could have easily used Amazon Simple Storage Service \cite{amazons3}
or any other distributed storage environment. Furthermore, Swift is an open
source application, but many users do not have access to or do not want to use a
Swift cluster. Other distributed storage services may be more convenient for a
user, physically closer, or provide guarantees (security, availability, or
otherwise) that a user highly values. Taking into consideration the diverse
selection of concerns and distributed resources, an application normally
chooses the best set of resources to address their concerns, often resulting
in having to interface with multiple different APIs.

% \section{Idea}

Here I present the Sage filesystem. Sage is a lightweight Unix-like filesystem
abstraction on top of backend storage devices. Instead of providing a
heavyweight client-server system, Sage is designed to sit on top of existing
storage backends introducing minimal overhead while providing a common
interface for applications to use. A key point of Sage is that it can abstract
any number of backends into a single usable filesystem under a common API. A
user does not need to use backend specific APIs. Instead, they use the Unix-
like Sage interface to store files remotely in independent systems. Sage works
over the wide area so it can aggregate backends distributed across the globe
into a single filesystem. Different backends have different characteristics
such as location, robustness, and security to list a few, so to some
applications, placing files in a specific backend may be very important. Sage
provides transparency into the system that allows users to access individual
components of the filesystem. Much like how each filesystem mounted on a Unix
system is still addressable, Sage allows users to address specific backends
individually within the system. If users do not want to choose a specific
backend or simply do not care then Sage takes over and places the file for
them. This way Sage provides an API to aggregate backends into a single
distributed filesystem, but still offers the flexibility for applications to
control file placement throughout the system.

This raises a few questions about what Sage can do:
\begin{itemize}
\item Can the system be made transparent enough to give users control over the location of file data on a per file basis?

\item Can the system provide enough flexibility so users can access many different remote storage platforms?

\item Can the system be made to scale while providing aggregate storage to multiple backends?

\item Does aggregating storage introduce significant overhead to the system compared to using resources separately?
\end{itemize}

The remainder of the dissertation is organized as follows, Chapter
\ref{chapter:rel_work} gives background and related work on distributed
filesystems and filesystem concepts. We look at central and distributed
filesystem management as well as aggregation in distributed filesystems.
Chapter \ref{chapter:arch} gives the architecture of Sage. I outline the
design goals of Sage and decisions on what distributed features (such as
replication, and consistency) Sage provides. Chapter \ref{chapter:imp} details
the prototype implementation of Sage. I discuss each Sage component and how it
interacts with the system, as well as how applications can use and take
advantage of Sage. Chapter \ref{chapter:exp} gives experimental results with
Sage using two backends Swift and MongoDB \cite{mongo}. The results report on the
performance and scalability with microbenchmarks, a case study looking at file
placement, and finally an examination on what applications make good use of
Sage versus those that do not. Chapter \ref{chapter:conc} has concluding
remarks as well as some directions for future work.
