\startchapter{Related Work}
\label{chapter:rel_work}


\section{Filesystem Concepts}

Before we dive into the background on distributed filesystems, let's take a small aside to define some common filesystem terms and concepts. A filesystem is traditionally an abstraction over some storage device used to store data. A filesystem partitions a device's available space into many blocks. Blocks normally have a fixed size (which varies from filesystem to filesystem) and are the atomic unit of most traditional filesystems. A file is stored as a collection of blocks, but there is a problem here. Files vary in size while blocks have a fixed size. So either the block size has to be sufficiently large so we can fit all reasonably sized files into a single one or we break up a file into a collection of blocks. If we use the former case, this means blocks must be large enough to store 1GB files. If we ignore the issues of handling 1GB buffers, the large block size means a 1B file will use the same amount of space as a 1GB file!

Using multiple blocks allows us to represent files of arbitrary size without much wasted space, however now we have to consider how we keep track of file blocks. As a user, we do not want to have to remember how many blocks an individual file has or where they are located to access a file. As a result, we want to store some information about the blocks. This information is known as file metadata and is traditionally stored in a structure called an inode. Inodes store metadata about files in a filesystem, not only where blocks are located but things like access times, permissions, and total file size. 

Using inodes and blocks we can describe files in a filesystem, but we still need to be able to describe the filesystem as a whole. Filesystems reserve some space for metadata about the entire filesystem itself that describes which blocks are free, where the root of the filesystem is, and other data about the filesystem layout. Filesystems are structured as a tree, with directories as nodes and actual files as leaves. The root of the directory tree (also known as the directory hierarchy) is called the root of the filesystem. Now that we have a general idea about what a normal filesystem does let's examine what distributed filesystems are. 


\section{Distributed Filesystem Key Ideas}

Distributed filesystems have an abundant history in computer science. The idea that resources could be accessed over the network has great traction as it gives machines access to a potentially enormous wealth of information, not available on a single disk. One of the first successful distributed filesystems is The Sun Network Filesystem \cite{Sandberg1985}. Known as NFS it was developed to allow a user to access the same filesystem from multiple workstations and share files with other users. NFS relies on a client server architecture where a central server holds all filesystem data and metadata, and clients connect to the server to access the filesystem remotely. NFS uses a remote procedure call (RPC) protocol, also developed by Sun, to allow clients to perform operations on the server. 

Clients mount NFS locally interacting with their systems virtual filesystem (VFS) layer allowing clients to see the NFS mount as a local filesystem. Clients can cache reads and writes for files, but must specifically check for invalidation with the server. As an aside NSV v3 allows clients to use weak consistency for improved performance. However, this consistency model allows clients to use stale data, so a tradeoff exists for clients to consider \cite{Pawlowski1994}. Since the server handles all client transactions, it can perform file locking, which it does at the inode level (as opposed to individual file blocks) to avoid write conflicts. NFS works very well, but the central server becomes a bottleneck at high loads as it is a central component of the entire system. 

Another big player in distributed filesystems is the Andrew Filesystem (AFS). Originally developed for the Andrew computing environment at Carnegie Mellon University \cite{Howard1988,Howard1988a,Howard1985}, AFS takes a different approach than NFS. It tries to move away from the central server of NSF and spread out filesystem operation over many smaller servers. These ``file servers'' are each responsible for sets of files logically called volumes. Each file server can be responsible for multiple volumes, but each volume is managed by one server. File servers store access control lists to handle authorization and hand out file locks and authentication tokens to clients to handle file access. A location database holds a mapping of files to file servers, which is organized into logical paths. Each file server has a location database that, when queried, can either return the requested file if present or return the file server that holds the requested file. AFS uses client side caching to store files, when a client opens a file, a copy of it is sent to the clients machine and cached where it can be manipulated locally. The cached file is pushed back to the server when the file is closed. A client side cache manager handles cache consistency and filesystem namespace lookups. The cache manager keeps a copy of the filesystem directory tree so file lookups can be done without contacting the file servers. File servers are responsible for invalidating the cache managers contents including any cached files or filesystem structure. 

NFS and AFS demonstrate two different designs in distributed filesystems. NFS has distributed clients but one central component that deals with filesystem requests. This simplifies dealing with consistency and locking issues, but introduces a single bottleneck. AFS distributes requests over multiple servers but must now have some structure describing how to access files, in this case a location database, which must be managed by the system. These two ideas have been the major driving force behind distributed filesystem development. In the next sections, we survey the landscape of distributed filesystems. First we will take a look at distributed filesystems that use centralized management, then examine those with decentralized management, and finally visit some filesystems that aggregate resources together.


\section{Centralized Management}

New classes of applications that require high throughput access to file data, either reads or writes, has spurred developments for centralized filesystems. In this section, we examine filesystems with centralized management specifically their design decisions and goals.

The fast secure read only filesystem \cite{Fu2000} is a read-only filesystem
that focuses on availability and security. To achieve high availability it
uses replication of a central database. This database of files is created on a
single server, which is then replicated to other machines that actually serve
files to clients. Clients interact by mounting a modified NFS drive on their
local filesystem, which allows them to communicate with one of these replica
databases. Clients only have read access on files provided by the
replications. The system extensively uses hashing to ensure file integrity,
and whole filesystem integrity. The central system hashes file and its entire
directory tree which is then handed to the replicas. This ensures that clients
can verify the replica has not been modified by the server that is hosting it.
The filesystem is great for content delivery to many clients as the replicas
are all identical and the client library (the modified NFS mount) can find the
best (lowest latency in this case) replica database to make requests to.


TidyFS \cite{Fetterly2011} is specifically targeted at write once high
throughput parallel access applications. According to TidyFS files are
abstracted into streams of data, which means files are actually composed of
many parts. In TidyFS a part is the smallest unit of data that can be
manipulated, however the part size is not fixed and is actually controlled by
the client. When a client wants to write a file it first chooses which stream
to write in, if it chooses to write a new stream then it can then choose a
part size for the stream.  Each part in a stream is lazily replicated and
stored on multiple OSDs. It uses a centralized metadata server (MDS) to keep
track of which parts make up a stream, as well as where each part replica is
located. Part metadata is essentially a key value store mapping part name to
data in the MDS. Each part is written only once (called immutable) and is
given a time to live (ttl). When a parts ttl expires it is deleted. An updated
file is actually rewritten (at least the part that was updated), and the
metadata server is updated to ignore the old parts of the file. To access a
file, a client contacts the MDS and is given the location of the closest up to
date replica, which it then reads directly off of the OSD or set of OSDs if
multiple parts reside in different locations. A virtual directory tree is
implied by file pathnames, but no hierarchy actually exists. Since the
filesystem is highly coupled to the MDS, it too is replicated, and decisions
are made based on the Paxos algorithm. Clients communicate with the MDS (or an
MDS in this case) through a client library which can help with load balancing
by directing to any replica of the MDS. Parts are replicated lazily (implying
replication does not block file access to the original) and placed
pseudorandomly on a set of machines, trying to choose machines with the most
available storage. The replica placement generates a set of three machines
based on the name of the replica to place, then chooses the machine with the
most available storage to place the file. If this is not done then small parts
may get all replicas placed on the same machine, which defeats the purpose of
replication. An interesting feature of TidyFS is clients can actually query
for part location to discover where the part physically exists.


The EU DataGrid Project (EDG) \cite{Kunszt2005}, \cite{Cameron2004} uses
Reptor \cite{Kunszt2004} for replication management. Reptor uses replicas to
improve file availability to grid applications, and uses a central service to
keep track of replicas of files. When a client asks for a file Reptor finds
the best replica, and sends the location back to the client. Reptor is
implemented as a collection of modules (called services), which interact
together to provide replication, consistency, and security to the EDG. Having
different services allows Reptor to be extended and easily customized to fit
the desired workload and application. The remaining services in the EDG host
the central replica catalog service, as well as the replica optimization
service called Optor. Optor can gather network information about the grid and
decide which link should be used to transfer a file stored in Reptor. When a
file is to be replicated a request is sent to the Replica Manager (Reptor),
which then contacts a Replica Metadata Catalog. The catalog translate the
logical file name into a unique identifier and sends it back to the manager.
The Replica Location Service is then contacted to find all replica locations
for the file identifier. The locations are then passed to the Replication
Optimization Service (Optor) to choose the best location for the new replica.
The file is then replicated to the new location and the new copy is registered
with the Replica Location Service.


The Panasas ActiveScale Storage Cluster \cite{Nagle2004} is a cluster storage
system with a central MDS, and many OSDs (object storage devices) which
actually store files. Panasas uses an abstraction it calls an object to store
file data. Objects contain file data as well as some metadata about RAID
parameters and data layout and things normally found in an inode such as file
size and data blocks. This allows the OSD to handle each object differently
and manage some metadata of the file.The MDS is responsible for the filesystem
structure which points clients to the OSDs where the file contents are stored.
Files can be striped in multiple objects over multiple OSDs. To do this the
MDS holds a map of each file which describes where the file components are
located. The MDS also handles client cache consistency. Clients are allowed to
cache file maps, but it is the responsibility of the MDS to tell a client if
its data is stale invalidating the cache. The last thing the MDS is in charge
of is file access. It hands out capabilities to clients (which can be cached)
that describe what a client is able to do to a file. Since capabilities can be
cached it is the MDS which must invalidate the capability when needed. Apart
from caching file maps and capabilities the OSDs can cache writes and reads.
Panasas has specific hardware requirements that take advantage of hardware
disk caching to improve file throughput. Finally as perviously mentioned all
metadata not related to the MDSs functions (which are mainly directory
structure and file access) is stored with the file itself on the OSDs. This
along with caching of the file map can allow many metadata operations to
bypass interacting with the MDS thus alleviating load. Clients interact with
Panasas through a kernel module which allows the filesystem to be mounted on
the clients machine.


XtreemFS \cite{Hupfeld2008} is a filesystem that like Panasas also uses the
object abstraction for files, and attempts to improve grid filesystem
performance using file objects.  The filesystem is partitioned into volumes
which are completely separate from each other, have their own directory
structure, and their own set of access and replication policies. An overall
metadata service (called the directory service) in a central server handles
organization of volumes as well as structures called metadata and replica
catalogs (MRCs). MRCs hold all the metadata for a set of volumes in a database
which has an internal replication mechanism, which can replicate data to other
MRCs. This means any given volume can be present in more than one MRC (the
volumes metadata simply has to be in the MRCs database). A volume has a set of
policies which allows the MRC to control the consistency of the replicas
differently in each volume. This allows XtreemFS to have volumes with
different policies that restrict placement of files (or replicas in this case)
to a specific set of OSDs. The directory service connects clients to MRCs and
is the only centralized component of the filesystem. A client interacts with
an MRC which describes the volumes where the actual data resides, which the
client can then contact to perform operations on. Volumes physically reside on
OSDs. Consistency of a file object is handled by the containing OSD, not the
volumes MRCs. When given a request the OSDs act in a peer to peer manner with
other replica holders to serve a file and maintain consistency. OSDs also
maintain leases for files which along with version numbers for files helps
maintain consistency, and resolve data conflicts.


Lustre \cite{Microsystems2007} is a distributed filesystem that in 2003 ran on
three of the eight largest clusters in the world \cite{Schwan2003}. It uses a
centralized MDS to handle metadata with many client object storage servers
(OSSs) which store actual data. Data is grouped into logical volumes
maintained by the MDS which are then seen by clients like normal filesystems.
A standby MDS provides redundancy in case the active MDS encounters a problem
and all requests done on the active MDS are done on the standby as well. Files
are represented as a collection of objects on the MDS, which are physically
stored on the OSSs. Objects belonging to the same file can be stored on
different OSSs to provide parallel access to parts of a file (called object
striping). Lustre uses file locking to ensure file consistency through its
distributed lock manager \cite{Schwan2003}. The lock manager is a centralized
component that grants locks to distributed clients. Locks can be read, write,
and some interesting variations that allow clients to cache many operations to
lower communication costs between the MDS and the client. In really high
contention spots in the filesystem (such as /tmp) the lock manager will not
give out a lock, and will actually perform the clients operation itself. This
avoids having to pass a lock back and forth rapidly. Clients are actually able
to cache the majority of metadata operations locally and only have to check
consistency when a new lock is requested.


GPFS \cite{Schmuck2002} is a large filesystem which uses unix like inodes and
directories, but stripes file blocks over multiple storage nodes to improve
concurrent access to the file. File blocks are typically 256kB and a single
file may be striped over multiple nodes with block placement determined in a
round robin format around the nodes in the filesystem. Every file has a
metanode which is somewhat equivalent to a standard filesystem inode and
contains the locations of all the blocks of the file. A single node in the
system known as the location manager handles allocation of new space on other
nodes in the filesystem using a map structure that identifies unused space. To
achieve high throughput and ensure consistency GPFS uses distributed file
locking. A central lock manager is responsible for handing out smaller locks
for parts of the filesystem. These smaller locks can be broken up into even
smaller locks by the files metanode, all the way down to byte range sizes on
files. By locking down to byte range granularity GPFS can easily support
parallel file access to the same file. All metadata updates to nodes are
handled by the metanode. Other nodes will update metadata in a local cache
then send the contents to the metanode which pieces the updates together. GPFS
does not replicate files, instead it uses a RAID configuration. GPFS can also
run in a mode if POSIX semantics are not needed for the filesystem. Called
data shipping mode, no locks are handed out and instead, nodes become
responsible for specific blocks of data. When operations are performed on the
data, the request is forwarded to the handling node and carried out on it.


PARTE  \cite{Liao2012} is a parallel filesystem that focuses on high
availability through an active and standby metadata server, as well as
metadata striping. PARTE uses a central MDS to handle file requests and
several object storage servers which it calls OSTs. When a client wants to
perform an operation on a file it first contacts the MDS, which then grabs the
inode of the requested file, and updates the inode metadata with unique client
and log ids and the file version number if needed. The inode is then written
and the metadata response is sent back to the client which can then perform
operations on the file. The MDS replicates stripes of its metadata on OSTs to
improve availability and allow the MDS to recover in case of failure.
Synchronization of metadata on the OSTs is done by the client and log ids that
are stored with a file, along with the version number. In fact if an MDS is
recovering from failure (said to be in recovery mode), the OSTs holding
metadata can process metadata requests from client admittedly at a slower
rate.


The Google File System (GFS) \cite{Ghemawat2003} was developed by Google to
support distributed applications. Typical google applications are large scale,
require built in fault tolerance and error detection, automatic recovery, and
deal with multi gigabyte files. An example of such an application is Bigtable
\cite{Chang2008}, which is a large key-value store system where data is
addressed by a key. A key is composed of identifiers including a columns key,
row key, and time stamp. Bigtable provides very fast access to data as it is
essentially a sorted map, of all the keys and values, but ultimately stores
data in GFS. Googles goals were to support many large files of 100MB or more,
with files being written to a small number of times, and read a large number
of times. Additionally files are mostly appended to rather than randomly
written to, and reads are usually large sequential reads. To meet these goals
the GFS provides a POSIX (Unix) like interface, with files referenced in
hierarchical directories with path names, and supports create, read, write,
amd delete operations. Interestingly the GFS implements an atomic append
operation, which helps simplify locking on files. GFS has a single master that
stores metadata for the entire filesystem and multiple chunkservers that store
data. Files are broken up into chunks of 64MB which are replicated (to avoid
using RAID and still provide data durability) and stored on chunkservers. The
metadata stored for the cluster includes namespace information like paths,
permissions, and mappings from files to chunks as well as location of the
individual chunks. Applications interact with the GFS through client code
which implements a file system API, but does not go through the operating
system. When performing operations on files, clients interact with the master
to get the appropriate chunkservers, then interact directly with the
chunkservers to access data. All metadata is maintained in RAM by the master,
but is also flushed to disk periodically. It does not flush chunk locations to
disk however. In case of a failure the master asks each chunkserver which
chunk they have which alleviates the need of the master to verify the
locations of all chunks. The master also contacts each chunkserver
periodically through a heartbeat message through which it can collect the
chunkservers state. File locks are done via read and write leases, on a per
file basis given out and maintained by the master. When files are deleted they
are not immediately reclaimed, instead they are marked for garbage collection,
which is then done by the master. No caching of file data is performed on
clients, as typical workloads require data to large to be cached, but chunk
locations can be cached. Although clients still need to contact the master for
leases if they have expired.


TLDFS \cite{Wang2007} is a layered distributed filesystem consisting of a
block device layer, which handles where actual data blocks reside, and a
system layer which handles locking and communication between different
filesystem components. The block device layer aggregates all the physical
storage of the nodes in the filesystem and makes it appear as one large
resource (when it is in fact a pool of smaller resources). This layer is
responsible for converting logical addresses from the system layer into
physical addresses of individual machines. The layer also sends out heartbeat
messages to all connected storage machines in order to keep track of who
remains in the filesystem. This allows machines to attach dynamically without
having to notify the system level of the filesystem. The system layer manages
filesystem components in both userspace and kernel space of client machines.
Each node has lock server which is used to manage consistency. The lock server
maintains queues of locks on individual inodes (called blocks) within the
filesystem with a given lock server responsible for a set of blocks. Locks are
either read or write, and have the classic multiple reader one writer
semantics. When a client writes a file, it acquires a write lock, performs
file modifications in a local buffer, then flushes the buffer back to the
server when the lock is released. The filesystem layer also contains an
interconnect module which contacts all other client nodes within TLDFS using
heartbeat messages. The interconnect module allows client nodes to request
locks from the lock manager present on others, and therefore manipulate files
maintained in other parts of the filesystem.


The Hadoop Distributed File System (HDFS) \cite{Shvachko2010} is an integral
part of the Hadoop Map Reduce Framework, and was created to service the need
for large scale MapReduce jobs. HDFS is designed to support a very large
amount of data distributed among many nodes in a cluster and provide very high
I/O bandwidth. HDFS consists of a single NameNode that acts as a metadata
server, and multiple DataNodes which store file data. DataNodes are used as
block storage devices, and do not provide data durability with RAID. Instead
data is replicated on different DataNodes distributed across the filesystem to
provide durability in case of node or disk failure. In addition to providing
robustness distributing data also increases data locality in HDFS. Locality is
a unique design goal of the HDFS as storage nodes are also frequently running
MapReduce jobs and high data locality improves the latency of transferring
data. The NameNode stores metadata for files and directories in an inode
structure which, like in a normal filesystem, store permissions, access times,
namespace, and other such attributes. The NameNode also stores locations of
file replicas as well as the directory tree. The directory tree is all kept in
main memory and periodically written to disk at a checkpoint. A journal is
kept of operations performed between checkpoints so the NameNode can recover
by taking the last checkpoint and replaying the journal. Much the opposite of
Googles GFS the DataNodes send heartbeat messages to the NameNode to ensure
they are still reachable. HDFS is not mounted in a normal Unix fashion,
instead clients interact through the filesystem Java api which supports
create, read, write, and delete operations. The clients are also exposed to
the physical location of files so the MapReduce framework can schedule jobs
close to data. File locking is done by acquiring read and write leases on
files from the NameNode. The leases are essentially locks that time out after
a given period of time.


MooseFS \cite{CoreTechnologyDevelopmentandSupportTeam2014} uses a central
server to store metadata and multiple chunk servers to store data much like
Google's GFS and HDFS.  Data is replicated on chunk servers and can be set per
file. MooseFS uses other metalog backup servers to log metadata operations and
periodically grab the metadata out of the central MDS, much like the
checkpoints done in HDFS. Clients interact with MooseFS through a FUSE module,
mounted on their local system.




\section{Distributed Metadata Management}


Deceit \cite{Siegel1990} is a filesystem that extends NFS. Normally to access
a given server a client has to mount the NFS server locally, in Deceit as long
as a client has mounted the Deciet filesystem, then they have access to all
servers mounted within. Each server still must be mounted to a client, but
servers communicate with each other and propagate information between them. In
other words the actual client only has to contact one server in the set of
servers provided by Deceit to access the entire filesystem, while in NFS the
client would have to mount each server separately. Deceit replicates files
over the set of servers and has a single write lock on each file. A file can
only be updated by the server when it has the write lock for the file.


In the Echo \cite{Hisgen1989} distributed filesystem the directory structure
is maintained in two parts. The upper levels of the directory tree (ie. the
root and directories close to the root) are described in a global table called
the global name service. The lower levels of the tree are each handled by a
separate server, so a server is responsible for a given subtree of the entire
filesystem. The servers that store data are replicated, but there is an
arbitrarily designated primary node that handles requests on a given file. The
primary takes a majority vote of all the file replicas to ensure it is serving
the correct version. Clients can cache files for quick access and it is the
primaries responsibility to notify the client if the cached copy needs to be
invalidated. The global name service is also replicated, but has weaker
consistency than replicated files. When the global name service is updated
updates are propagated to all replicas but service does not stop. This implies
that clients may contact an older version of the global table and can get two
conflicting answers from two different tables, however upper level directories
are modified much less than the leaves of directory trees.


Tahoe \cite{Wilcox-O'Hearn2008} is a distributed metadata filesystem with
emphasis on file security. Files and directories (as metadata are just files
in Tahoe) are distributed throughout hosts in the filesystem using erasure
coding. As an aside erasure coding is a way of encoding data which is very
failure resilient. Erasure coding takes a message with K symbols and expands
it to N symbols, N = K + M, where M are redundant symbols. To reconstruct the
message from N we only need K symbols out of the N.  Tahoe uses the two
erasure parameters N, the number of hosts a file is distributed to, and K, the
number of hosts required to be available for the file to be available. This
way Tahoe can distribute files over N hosts but only require K of them to be
available to recover a file. Tahoe also heavily encrypts data with AES and
uses SHA256 signatures to ensure data integrity. Individual files have
capabilities stored with them which address what clients can do (or not do) to
files.


Group-based Hierarchical Bloom filter Array (G-HBA) \cite{Hua2007} is a scheme
to manage distributed metadata using bloom filters to distribute metadata over
a number of Metadata Servers (MDSs). Bloom filters are structures which can be
used to check if an element is a member of a set. While space efficient, bloom
filters are probabilistic so they can not be certain a given element is a
member of a set, however they do not produce false negatives (only false
positives) so they can be used to determine if an element is not in a given
set. G-HBA uses a group of MDSs to hold file metadata where a single given MDS
is responsible for a set of files. A file that a given MDS is responsible for
is called the files home MDS. Each MDS hold arrays of bloom filters which
point to other MDSs so when a file is queried at a given MDS, if the MDS is
not the files home, then the request gets forwarded to another MDS predicted
by the bloom filter. Clients can therefore randomly choose an MDS to query for
any file as they will get forwarded to the files home MDS.


Gluster \cite{Gluster} is a filesystem that has no metadata server. Metadata
is stored with a file which is located by an elastic hash function. Little
information is present on the Gluster created elastic hash function, however
the idea boils down to hashing files over a set of resources. This means Hash
values are used to place files on a set of logical volumes within Gluster.
When a client requests a file, they hash the path of the file to determine
which logical volume the file resides on, they then consult a map to find
which physical server to contact. Volumes are in fact replicated so a given
file is also replicated over all servers responsible for the volume it belongs
to. Not having a metadata server removes a single point of failure in the
system, but also makes it so that last write wins in consistency semantics, as
there is no watchdog over how many clients are reading and writing a given
file. Clients can mount Gluster filesystem through a FUSE module. OpenStack
Swift \todo{cite{SWIFT????} } works in a similar way using a hash function to
partition data over nodes. Swift however is an object storage system and
clients interact over a REST interface to contact the storage system.


BlobSeer \cite{Nicolae2011} is a filesystem heavily based on versioning to
provide consistency and concurrency. The architecture consists of: several
storage servers, one storage service which is queried to find free space,
several metadata servers, and one version manager which keeps information on
file snapshots. A main concept in BlobSeer is that data is never modified, it
is only added and superseded. Data is written in chunks, which receive a
unique chunk id and are striped over storage servers. files are described by
structures called a descriptor map which list the set of chunks that belong to
a specific file. These descriptor maps also receive a unique id and are stored
in a global map. Versioning is then done by addressing a specific descriptor
map, which in turn addresses specific chunks, and since data is never deleted
we are always guaranteed to find the correct version of the file pointed to by
the desired descriptor map. A file can have many descriptor maps and the maps
along with the related chunks are referred to as a snapshot of a particular
file. Like file data Metadata is never deleted either. Metadata for a file is
stored as a distributed segment tree, where each branch of the tree is
responsible for a different segment (byte range) of the file (or snapshot of
the file in this case). Descriptor maps belonging to the specific byte ranges
are stored with the leaves of the tree, so to get the correct maps required
for an operation the tree is walked returning the descriptor maps at the
resulting leaves. The segment trees are stored in a global structure
distributed over all metadata servers along with all other global structures.


The ideas presented by \cite{Blaze1992} aim to utilize client caching to
reduce load on filesystem servers. Here caches are used to store data on
clients, but if there is a cache miss clients are allowed to look in other
clients caches for the desired data. To do this cache hierarchies are
constructed either statically or dynamically. In a static hierarchy a
determined set of clients are contacted in case of cache misses (usually in
multiple layers), while in dynamic hierarchies they are built on the fly.
Clients can cache heavily shared files up to a certain number of copies. Once
this number has been reached the server hands out a list of clients with a
cached copy of the requested file to the requesters. The requester can then
choose from the list of cached copies to read the file, and can keep the list
of clients with a real copy cached.  Cache invalidations are propagated the
same way from the server to the set of machines caching the file, then from
those machines to the next in the hierarchy. In this sense each node can act
as a mini server for a file where other nodes can read a file from its cache
and invalidations are passed the readers when necessary.


zFS \cite{Rodeh2003} is a distributed file system design with a traditional
Unix like interface. It uses object storage to store files, but does not
distinguish between directories and regular files. zFS is designed to support
a global cache to improve performance. Files and directories are stored as
objects on storage servers. Directories contain pointers to other objects,
much like a directory in Unix storing inode numbers, and results in metadata
being stored with files much more like a traditional Unix file system. The
metadata for an object does not have to be placed on the same node so object
lookups take place separately from object reads and writes. zFS clients can
directly access objects once a lookup has been done. No replication is done by
the filesystem, instead data durability is left to the object store to handle
(either RAID, or replication at the object level). Each node in zFS is
responsible for objects located on it, and generates leases when an object is
to be read or written by a client. zFS keeps a global cooperative cache, which
exists in memory on each machine. The observation is that it takes less time
to fetch from other machines memory over the network, than it does through the
local machines disk. When an object is requested it is first searched for in
the cooperative cache for all machines. If it is found it can be read from the
cache rather than where it is stored on disk. The cache is managed for
consistency, and only data that is not being modified on other hosts
(queryable via leases on the object) is cached which provides strong cache
consistency.


xFS \cite{Anderson1996} distributes management of the filesystem with metadata
managers, and storage servers. The metadata managers hold metadata for the
filesystem, while the storage servers hold actual data. Additionally all
clients participate in a global cache to provide high data availability.
Metadata is distributed according to a Manager Map, which is globally
replicated on all clients and servers. The Manager Map is essentially a table
that maps groups of files to specific metadata managers and can be updated on
the fly. Metadata managers contain collections of imaps, which describe which
storage server a file resides on, where the file is located on disk, and the
location of all cached copies of the file. Any given file is represented by an
index number. Looking up a file in a directory returns the index numbers of
the files contained within, which can then be used to find the desired files
manager, which is then used to get the imap and access the file. Portions of
files are striped across many storage servers by grouping files into stripe
groups. If a file stripe exists on a given storage server, then it will exist
on all storage servers in the stripe group. Stripe groups are identified by a
Stripe Map, which is globally distributed throughout the filesystem. Managers
are responsible for file stripe consistency and keep track of all cached
copies (seen before in the imap). When a stripe is updated the manager must
invalidate all cached copies of a stripe and update the stripes imap.


The authors of \cite{Triantafillou1997} lay out a set of protocols for high
replication in distributed filesystems where files are replicated at multiple
servers. Clients are allowed to cache files, but before an operation is
performed they query a set of servers to see if their copy is up to date. The
servers will check all replicas of the file queried, and return the most up to
date version of the file (based on majority), and inform other replicas that
they are now obsolete. If a file is to be modified a timestamp is generated
and updates to the file are serialized according to the timestamps in a write
queue.  This ensures that all up to date replicas have applied the updates in
the same order.


JigDFS \cite{Bian2009} is a distributed filesystem with a high emphasis put on
security. Much like Tahoe, JigDFS splits up files using erasure codes and
stored on multiple machines, however the erasure codes are used iteratively
and with a hash chain to avoid information leakage. To find all the parts of a
given file a chain of hash values each depending on the previous result is
used. A distributed hash table keeps track of where files are located (at
least to start the hash chain), which is globally maintained by the nodes of
the system. Nodes act in a peer to peer manner maintaining files in the
filesystem. Each node is responsible for the parts of files stored there, and
a portion of the distributed hash table.


Coda \cite{Satyanarayanan1990} is a distributed file system with the overall
goal of constant data availability, and takes a different approach than the
previously examined file systems. Coda uses a few trusted servers to handle
authentication, but allows clients to aggressively cache data. Coda also uses
server replication to provide high availability. A client uses a working set
of servers for file system operations, and is said to be connected if it can
communicate with at least one of the servers. While connected files are pushed
to the servers from a local cache when mutated. If a client loses its
connection to all of the servers it starts operating in disconnected mode, and
operates solely out of its local cache without pushing changes. When the
client reconnects to a server, it pushes the local cache to the file system.
Coda uses an optimistic replication strategy, meaning it pushes changes from
the cache without knowing the files state the in the file system. Coda
provides conflict detection to identify when a file is updated on two separate
clients. If the files modifications do not conflict, Coda automatically
resolves the conflict, otherwise a new file is created and the conflict must
be resolved manually. Interestingly Dropbox takes the same approach to
resolving conflicts and disconnected operation \todo{cite{DROPBOX???}}.


DNM \cite{Wei2000} attempts to distribute metadata namespace over metadata
servers (called DNM servers) using a global table. The table is globally
replicated and contains the root and the first level of subdirectories (much
like Echo), the rest of the namespace is partitioned over metadata servers
into subtrees, which are then handled independently by DNMs. The global table
holds a mapping of directory to the appropriate DNM server so when a client
makes a request to the filesystem it queries a server which will look up the
correct server in the name table and forward the request to it. Clients
aggressively cache lookup results and the client caches not only the final
result of the lookup, but all intermediate directories in the request. This
creates a tree like cache on the client which it can then use to facilitate
further requests to files that share a portion of past ones. DNM servers hold
file locations, which again can be cached, and are revalidated when a lookup
fails on file serving nodes.


DMooseFS \cite{Yu2012} aims to distribute metadata around MooseFS using
multiple independent metadata servers to host filesystem metadata. Each MDS is
responsible for only a portion of filesystem metadata. The directory structure
is distributed among the metadata servers using a hash table. When a client
sends a request to the filesystem, the path of the file is hashed which will
determine which MDS the request is sent to. The MDS then tells the client
which set of chunkservers to contact for the file data. The directory
structure is only partially hashed (much like how Echo and DNM split up the
directory hierarchy) so an MDS is responsible for a given subtree of the
directory structure, as each MDS is oblivious to others.


Ceph \cite{Weil2006} relies on metadata nodes and storage nodes to provide a
distributed file system, and maintains them as two clusters, a metadata
cluster and a storage cluster. Clients interact with the metadata and storage
clusters separately to perform operations. Metadata for the cluster contains a
mapping of files to locations as well as other file metadata (size, etc), but
to locate a file a distribution function is used. Any entity that knows the
distribution function can compute where in the storage cluster a file is
located. A hash function is a simple distribution function used by Gluster and
Swift, but erasure codes like in Tahoe and JigDFS can also be used. This
eliminates object lookups for locating files, however a lookup is still
required to manipulate a file's metadata. Ceph distributes the metadata in a
cluster as a hierarchy, where a given server is responsible for a portion of
the filesystems structure. The portion of the filesystem each metadata server
is responsible for can be dynamically updated, which allows flexibility and
load balancing in the metadata cluster. MDSs hand out capabilities to clients
that allow them to read and write files from the storage servers OSDs. Files
are replicated and distributed over the OSDs using the CRUSH algorithm. CRUSH
or Controlled Replication Under Scalable Hashing \cite{Weil2006a} is an
algorithm for file placement specifically developed to place object replicas
in a distributed environment. CRUSH takes an object identifier as input (could
be a path name or id) and outputs a list of storage devices to place the
replicas. CRUSH tries to optimize replica placement according to assigned
storage device weights, where a more heavily weighted device will end up with
more objects (well, more replicas of different objects). For CRUSH to work it
needs to know about the storage cluster layout, the weights of each node in
the cluster, and makes use of a mapping function to essentially hash the
object identifiers. Looking back at Ceph each OSD stores data locally in an
Extent and B-tree based Object File System (EBOFS) which supports atomic
transactions (writes and attribute updates are atomic) and allows Ceph to take
control of the physical machines block device. The storage cluster is directly
accessed by clients once they have file locations and capabilities to
manipulate files. Clients can interact with Ceph through client code either
linked into applications, or through a kernel module.


\section{Existing Filesystem Aggregation and other Concepts}

The user level secure Grid filesystem (SGFS) \cite{Figueiredo2007} modifies
the NFS protocol to use SSL to ensure secure communication in a grid
environment. It modifies NFS by adding proxies at the endpoints of
communication that encrypt NSF traffic using SSL. SGFS allows users to choose
between encrypting and digitally signing messages for security or just
digitally signing to improve performance over the former. Additionally the
protocols used to encrypt and sign messages can be chosen by the user.


The InterMezzo \cite{Braam1999} filesystem is a layered filesystem that
organizes file sets into logical volumes. An entire file set resides on a
single server and clients mount individual volumes onto their system. A
central database described which server a volume resides on. Clients can mount
multiple volumes to create a local directory tree. Any mounted volume can be
the root of the clients filesystem, and other volumes are mounted inside the
root. Metadata for file objects are stored with the files themselves which
makes volumes very similar to local filesystem volumes. When an object is
updated a permit must be acquired for consistency, which then allows the
update to be propagated from the updating client to the server. Clients cache
data and are allowed to operate on the cached data while it is still fresh.
The cache is managed by a separate process (called Lento) that communicates
with the server of the cached file set.


The Trivial Distributed Filesystem (TDFS) \cite{Voras2006} is a simple
distributed filesystem aiming to implement remote storage using a simple
client server model. TDFS consists of two processes, a master and a slave
process. The master process is mounted on a client system and attaches to a
slave process that is running on a remote host. The master forwards operations
performed on the clients system over to the host the slave is running on
blocking until the operation has completed. A master may only connect to a
single slave process, therefore to mount multiple remote machines multiple
master processes have to be run, creating multiple mount points on the client
system. The slave process is also only connected to one master process.


IncFS \cite{Zhao2006} creates a distributed filesystem by combining many NFS
deployments into a single filesystem. A single NFS server is designated as the
meta server which stores all the metadata information about the filesystem,
and the remaining NFS deployments store actual data. IncFS is implemented
through a virtual filesystem layer which intercepts all independent NFS mounts
and combines them into a single mountable volume. The volume can be mounted by
any number of clients and appears just like a single NSF mount. Under the
hoods IncFS simply mounts all NFS instances and uses one as the metadata
server to translate logical filenames into physical ones actually present in
the other NFS mounts.


GMount \cite{Dun2009} allows users to mount directories from many remote
machines into a single local location. By using multiplexing and ssh, remote
connections are established to remote machines which transfer files over sftp
when accessed. Entire directory trees can be mounted on multiple clients using
GMount, which uses last write wins semantics to handle conflicts. The
architecture is more of a peer to peer model in the sense every machine can
mount directories from each other. No caching is done by clients.


Chirp \cite{Donnelly2012} is a user level distributed filesystem that allows
the aggregation of many other filesystems to be mounted as a single entity.
Clients mount the chirp filesystem locally and interact with the Chirp server.
The Chirp server is a centralized component that handles requests from all
clients to the Chirp filesystem. The server forwards client requests to
containing filesystems, managing access control lists on files and
authentication with Chirp itself. Chirp is very concerned with authentication
and does so by passing around authentication tokens to make sure clients can
only access data they are authorized for.


Cegor \cite{Shi2004} is an NFS like filesystem, where clients interact with
servers to handle both file data and metadata requests. Connections in Cegor
revolve around the notion of semantic views. Normally connections are handled
through the TCP/IP stack of the server, but in Cegor both clients and servers
store information that allows communication even when network connections
disconnect and reconnect with a different TCP connection. This allows clients
to move between networks and not lose connection to the filesystem, or have to
reconnect entering credentials again. The actual filesystem consists of an NFS
like server to serve files and communicate with clients. Clients are allowed
to cached data and take out read/write leases on files. If a client
disconnects, a reconciliation step happens where the client validates its
cache, and then performs the modifications on the new cache contents.