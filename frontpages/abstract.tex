\newpage
\TOCadd{Abstract}

\noindent \textbf{Supervisory Committee}
\tpbreak
\panel

\begin{center}
\textbf{ABSTRACT}
\end{center}


Modern distributed applications often have to make a choice about how to
maintain data within the system. Distributed storage systems are often self-
contained in a single cluster or are a black box as data placement is unknown
by an application. Using wide area distributed storage either means using
multiple APIs or loss of control of data placement. This work introduces Sage,
a distributed filesystem that aggregates multiple backends under a common API.
It also gives applications the ability to decide where to store file data in
the aggregation. By leveraging Sage, users can create applications using
multiple distributed backends with the same API, and still decide where to
physically store any given file. Sage uses a layered design where API calls
are translated into the appropriate set of backend calls then sent to the
correct physical backend. This way Sage can hold many backends at once making
them appear as the same filesystem. The performance overhead of using Sage is
shown to be minimal over directly using the backend stores, and Sage is also
shown to scale with respect to backends used. A case study shows file
placement in action and how applications can take advantage of the feature.
